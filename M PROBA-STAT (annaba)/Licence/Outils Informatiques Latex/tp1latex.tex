\documentclass[a4paper,12pt]{article}
\usepackage[utf8]{inputenc}
\usepackage{graphicx}
\usepackage[T1]{fontenc}
\usepackage{fancybox}
\usepackage{xcolor} 
\usepackage{amsmath,amssymb}
\usepackage{pifont}

%%%%%%%%%%%%%%%%%%%%%%%%%%%%%%%%%%%%%%%%%%%%%%%%%%%%%%%%%%%%%%%%%%%%%%%%%%%%%%%%%%%%%%%%
\newtheorem{re}{Remarque}

\begin{document}
\author{A.BENYOUCEF}
\title{Premiers pas avec \LaTeX}
\begin{center}
\ovalbox{\LARGE{Environnement math\'ematique}}
\end{center}
\shadowbox{1-Les formules math\'ematiques en \LaTeX}\\
\begin{enumerate}	
\item Toute formule math\'ematique doit \^etre \'ecrite entre deux $\$\cdots \$$.
\item Pour centrer une formule math\'ematique on \'ecrit $\$\$\cdots\$\$$.
\end{enumerate}
\section{Limites, Int\'egrales, D\'eriv\'ees Sommes et Produit}
\subsection{Symboles et utilisation}
\begin{itemize}
\item\textbf{Limite:} \$\textbackslash lim\$
\item\textbf{Int\'egrale:} \$\textbackslash int\$
\item \textbf{D\'eriv\'ee:} \$\textbackslash prime\$, \$\textbackslash partial\$
\item\textbf{Somme:} \$\textbackslash sum\$
\item \textbf{Produit:} \$\textbackslash prod\$
\end{itemize}
\subsection{Exemples:}
\begin{minipage}[t]{8cm}	
\textbf{Code source}\\
\$\textbackslash lim\_\{x\textbackslash rightarrow 0\}f(x)\$\\
\$\textbackslash int\_\{0\}\textasciicircum\{1\}f(t)dt\$\\
\$ f\textbackslash prime\$, \$\textbackslash partial f\$\\
\$\textbackslash sum\{k=1\}\textasciicircum\{n\}k\textasciicircum2\$\\
\$\textbackslash prod\_\{k=1\}\textasciicircum\{n\} f(k)\$
\end{minipage}
\hspace{0.3cm}
\begin{minipage}[t]{8cm}
\textbf{PDF}\\
$\lim_{x\rightarrow 0}f(x)$\\
$\int_{0}^{1}f(t)dt$\\
$f\prime$, $\partial f$\\
$\sum_{k=1}^{n}k^2$\\
$\prod_{k=1}^{n} f(k)$
\end{minipage}	
\section{Vecteurs, Normes}
\begin{itemize}
\item \textbf{ Vecteur:} \$\textbackslash vec\{vecteur\}\$
\item \textbf{Norme:} \$\textbackslash| .\textbackslash|\$
\end{itemize}
\subsection{Exemples:}
\begin{minipage}[t]{8cm}	
\textbf{Code source}\\
\begin{itemize}
\item \$\textbackslash vec\{u\}\$
\item \$\textbackslash|u\textbackslash|\$
\end{itemize}
\end{minipage}
\hspace{0.3cm}
\begin{minipage}[t]{8cm}
\textbf{PDF}\\
\begin{itemize}
\item $\vec{u}$
\item $\|u\|$
\end{itemize}
\end{minipage}
\section{Les nombres complexes}
\begin{tabular}{|c|c|c|}
	\hline Op\'eration & Syntaxe & Compilation\\
	\hline R\'eel & \$\textbackslash Re(z)\$ & $\Re(z)$\\
	\hline Imaginaire & \$\textbackslash Im(z) \$ & $\Im(z)$\\
	\hline Conjugu\'e & \$\textbackslash overline \{z\}\$ & $\overline{z}$\\
	\hline Module & \$\textbackslash left| z\textbackslash right| \$ & $\left| z \right|$\\
	\hline
\end{tabular}
\section{Parall\`eles et perpondiculaires:}
 $D\perp D'$ se code \$D\textbackslash perp D'\$\\
 {43} $D\parallel D'$ se code \$D\textbackslash parallel D'\$
\section{Probabilit\'es}
\begin{minipage}[t]{8cm}	
\textbf{Code source}\\
\begin{itemize}
\item\$ A\textbackslash cup B\$
\item \$ A \textbackslash cap B = \textbackslash emptyset\$
\item \$ \textbackslash setminus B\$
\end{itemize}
\end{minipage}
\begin{minipage}[t]{8cm}
\textbf{PDF}\\
\begin{itemize}
\item $A \cup B$
\item $A \cap B=\emptyset$
\item $A \setminus B$
\end{itemize}
\end{minipage}
\section{Les ensembles:}
\begin{tabular}{|c|c|c|}
\hline L'ensemble & La syntaxe & Compilation\\
\hline Les nombres r\'eels & \$\textbackslash mathbb\{R\}\$ & $\mathbb{R}$ \\
\hline Les nombres entiers & \$\textbackslash mathbb\{N\}\$ & $\mathbb{N}$ \\
\hline Les nombres complexes & \$\textbackslash mathbb\{C\}\$ & $\mathbb{C}$ \\
\hline
\end{tabular}
\section{Symboles math\'ematiques}
\begin{minipage}[t]{8cm}	
\textbf{Code source}\\
\begin{itemize}
\item \$ \textbackslash leq \$
\item \$ \textbackslash geq \$
\item \$ \textbackslash approx \$
\item \$ \textbackslash equiv \$
\item \$ \textbackslash neq \$
\item \$ \textbackslash subset \$
\item\$ \textbackslash in \$
\item \$ \textbackslash notin \$
\item \$ \textbackslash pm \$
\item \$ \textbackslash times \$
\item \$ \textbackslash infty \$
\item \$ \textbackslash forall \$
\item \$ \textbackslash exists \$
\end{itemize}
\end{minipage}
\begin{minipage}[t]{8cm}
\textbf{PDF}\\
\begin{itemize}
\item $\leq$
\item $\geq$
\item $\approx$
\item $\equiv$
\item $\neq$
\item $\subset$
\item $\in$
\item $\notin$
\item $\pm$
\item $\times$
\item $\infty$
\item $\forall$
\item $\exists$
\end{itemize}
\end{minipage}
\section{Les fl\`eches}
\begin{minipage}[t]{8cm}	
\textbf{Code source}\\
\begin{itemize}
\item \$ \textbackslash rightarrow \$
\item \$ \textbackslash leftarrow \$
\item \$ \textbackslash iff \$
\item \$ \textbackslash implies \$
\item \$ \textbackslash mapsto \$
\end{itemize}
\end{minipage}
\begin{minipage}[t]{8cm}
\textbf{PDF}\\
\begin{itemize}
\item $\rightarrow$
\item $\leftarrow$	
\item $\iff$
\item $\implies$
\item $\mapsto$
\end{itemize}
\end{minipage}
\subsection{Principes des commandes g\'en\'erant des fl\`eches}
\ding{43}Toutes les commandes finissent par le suffixe arrow.\\
\ding{43}Toutes les commandes commencent par des pr\'efixes qui indiquent la direction
\textbf{left}(gauche),\textbf{right}(droite), \textbf{up} (haut), \textbf{down} (bas).\\
 \ding{43}Le pr\'efixe facultatif \textbf{long} donne une fl\`eche longue.\\
\ding{43}La premi\`ere lettre mise en majuscule rend la fl\`eche double.
\section{Radicaux, Fraction, Exposant, Indices}

\begin{minipage}[t]{8cm}	
\textbf{Code source}\\
\begin{itemize}
\item \$ \textbackslash sqrt\{x\} \$
\item \$ \textbackslash frac \{num\}\_\{denom\} \$
\item \$ \textasciicircum\{exposant\} \$	
\item \$\_\{indice\} \$
\end{itemize}
\end{minipage}
\begin{minipage}[t]{8cm}
\textbf{PDF}\\
\begin{itemize}
\item $\sqrt{x}$
\item $\frac{f(x)}{g(x)}$
\item $x^{2}$
\item$ x_1$
\end{itemize}
\end{minipage}
\section{Les lettres greques}
\begin{tabular}{|c|c|c|c|c|c}
\hline $\alpha$ \$ \textbackslash alpha\$ & $\beta$ \$ \textbackslash beta\$ & $\gamma$ \$ \textbackslash gamma\$ & $\epsilon$ \$ \textbackslash epsilon\$ & $\eta$ \$ \textbackslash eta\$ & $\theta$ \$ \textbackslash theta\$ \\
\hline $\delta$ \$ \textbackslash delta\$ & $\nu$ \$ \textbackslash nu\$ & $\mu$ \$ \textbackslash mu\$ & $\xi$ \$ \textbackslash xi\$ & $\psi$ \$ \textbackslash psi\$ & $\rho$ \$ \textbackslash rho\$ \\
\hline
\end{tabular}
\begin{re}
	pour \'ecrire une lettre greque il suffit d\'ecrire \$ \textbackslash nom\_de\_la\_lettre. Dans le cas d'une lettre greque majuscule, il suffit d'\'ecrire la premi\`ere lettre en majuscule.
\end{re}

	
\end{document}
