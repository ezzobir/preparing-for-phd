\documentclass[25pt,a1paper]{tikzposter}
%%%%%%%%%%
\usepackage[utf8]{inputenc}
\usepackage{fontenc}
\usepackage[french]{babel}
\usepackage{float}
\usepackage{amssymb}
\usepackage{amsmath}
%%%%%%%%%
\usepackage{graphicx}
\graphicspath{{/home/darrin/Downloads}}
\usetheme{Rays}
\usepackage{multicol}
\begin{document}
\title{Problème de la chaleur à symétrie radiale }
\author{I.Djerrar, L.Alem, L. Chorfi}
\maketitle
\begin{columns}
% COLUMN 1
\column{.65}
%Block A
\block{Abstract}{
\bigskip
Dans ce travail on s'intéresse à un problème de la chaleur. On résout le problème direct qui servira à résoudre un problème inverse associé dans $\mathbb{R}.$
}
%Block B
\block{Problème direct}
{
\section{Position du problème}
On considère la problème de la théorie de la chaleur, on coordonnée radiale suivant
\begin{equation}
\begin{cases}
\frac{\partial u}{\partial t}=\frac{\partial ^2u}{\partial ^2 r}+\frac{1}{r}\frac{\partial u}{\partial r},\hspace{2cm} r\in\left( a,b\right), \quad t>0\\
u(a,t)=f(t) , \frac{\partial u}{\partial r }(b,t)=0 \hspace{2cm} t>0\\
u(r,0)=0, \hspace{3cm} r\in(a,b)
\end{cases}
\end{equation} 
Le but de ce travail est de résoudre ce problème par deux méthodes, analytique et approchée
}
%Block C
\block{Construction de la solution }
{
\subsection{Construction de la solution à l'aide de la transformée de Laplace }
Soit $f$  une fonction tel que $|{f(t)}| \leq Ce^{\sigma T}, \sigma\geq 0,$\\
La transformée de la place $F(s)=L(f)$ est définie par:
$$F(s) = \int_{0}^{+\infty}f(t)e^{-st}dt,  Re(s)\geq \sigma $$
La transformée inverse est donnée par\cite{Ditkine}
$$f(t)= L^{-1}(F)(t)=\frac{1}{2\pi i}\int_{\sigma-i\infty}^{\sigma+i\infty}F(s)e^{st}ds$$
}
% COLUMN 2
\column{.35}
\block{Solution exacte et approchée}
{
\includegraphics[width=\linewidth]{fig2.png}
}
\end{columns}
\block{Conclusion}
{
Le problème est résolue par une approche basée sur la transformée de Laplace directe et inverse. Ce problème s'est réduit à une équation intégrale de Volterra de première espèce avec un noyau très régulier.
}
\block{Bibliograhie}
{
\begin{thebibliography}{}
\bibitem{Ditkine} Ditkine V. Proudnikov A. Transformation intégrales et calcul opérationnel. Traduit du russe edition MIR.Moscow, 1978.
\bibitem{Herbin} Herbin R., Analyse numérique des EDP. ENgineering school,Marseille 2011.
\end{thebibliography}{}
}
\end{document}