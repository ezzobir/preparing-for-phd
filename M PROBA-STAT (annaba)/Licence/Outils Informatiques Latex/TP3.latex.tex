\documentclass[a4paper]{article}
%%%%%%%%%%%%%%%%%%%%%%%%%%%%%%%%%%%%%%%%%%%
%pour la gestion de francais et les accents
\usepackage[utf8]{inputenc}
\usepackage{fontenc}
\usepackage[french]{babel}
%%%%%%%%%%%%%%%%%%%%%%%%%%%%%%%%%%%%%%%%%
\usepackage{amsthm}
\newtheorem{montheorem}{Th\'eor\` eme}[section]
\newtheorem{madef}{D\'efinition}[section]
\newtheorem{monexemple}{Exemple}[section]
\def\dfrac{\displaystyle\frac}
%%%%%%%%%%%%%%%%%%%%%%%%%%%%%%%%%%%%%%%%%%%

\begin{document}
\noindent
\bf  {Module: ( TP Informatique: LATEX  )\qquad\qquad    2020}\\
\bf {Master 1,\ Math\'ematiques Appliqu\'ees}\\
\begin{center}
\bf {TP 3: Sections, th\'eor\`emes}
\end{center}
R\'ediger en \large {LATEX}\\
\noindent
\section{Fonctions holomorphes }
 \begin{madef}
 Soit $ U \subset C$
 un ouvert.
 On dit que $f:U \rightarrow C$ est
 \textbf{holomorphe (ou analytique complexe)} dans $U$ si $f$ est d\'erivable $\forall z_0 \in U$, c'est \`a dire que
 $$
 \lim_{z\rightarrow z_0}\frac{f(z)-f(z_0)}{z-z_0}
 $$
 existe et est finie. On note la d\'eriv\'ee par $f^\prime(z_0)$.
 \end{madef}

 \section{Equations de Cauchy-Riemann}
 \begin{montheorem}
 Soient $U \subset C$ un ouvert et $f:U \rightarrow C$ telle que, si $z=x+iy$,
 $$
 u(x,y)=Re f(x+iy), \ \ \ \ v(x,y)=Im f(x+iy)
 $$
 Alors les deux assertions suivantes sont \'equivalentes:\\
 \textbf{i)} $f$ est holomorphes dans $U$,\\
 \textbf{ii)} les fonctions $u,v \in C^1(U)$ et satisfont, $ \forall (x,y)\in U$, \textbf{les \'equations de Cauchy-Riemann}

 $$
 \frac{\partial u}{\partial x}=\frac{\partial v}{\partial y}, \quad \frac{\partial u}{\partial y}=-\frac{\partial v}{\partial x}
 $$
En particulier, si $f$ est holomorphe dans $U$, alors
$$
f^\prime(z)=\frac{\partial u}{\partial x}(x,y)+i\frac{\partial v}{\partial x}(x,y)=\frac{\partial v}{\partial y}(x,y)-i\frac{\partial u}{\partial y}(x,y)
$$
\end{montheorem}
\begin{monexemple}
La fonction $\sin z=\dfrac{e^{iz}-e^{-iz}}{2i}$ est holomorphe dans $C$ et sa d\'eriv\'ee est $\cos z=\dfrac{e^{iz}+e^{-iz}}{2}$
\end{monexemple}

\end{document} 
