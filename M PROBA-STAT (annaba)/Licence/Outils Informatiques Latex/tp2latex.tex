\documentstyle[12pt]{article}

%%%%%%%%%%%%%%%%%%%%%%%%%%%%%%%%%%%%%%%%%%%%%%%%%%%%%%%%%%%%%%%%%%%%%%%%%%%%%%%%
%\input texinput/a42.tex
%\input texinput/amatharti.tex
%\input texinput/mac.tex
\pagestyle{myheadings} \markboth{}{}

\def\dfrac{\displaystyle\frac}
\def\dint{\displaystyle\int}
\def\dsum{\displaystyle\sum}
\def\text{\hbox}
\def\iint{\int\!\!\int}
\begin{document}
\noindent
\bf  {Module: ( TP Informatique: LATEX  )\qquad\qquad    2020}\\
\bf {Master 1,\ Math\'ematiques Appliqu\'ees}\\
\begin{center}
\bf {TP 2: Formules math\'ematiques, tableaux}
\end{center}
R\'ediger en \large {LATEX}\\
\noindent
\small {\bf Excercice 1.}
Pour $n$ entier naturel, on pose $u_{0}$ et $u_{n}=u_{n-1}+n$. Alors\\
\begin{equation}
\forall n\in {N}^{\star} , u_n=\frac{n(n+1)}{2}\geq 0
\end{equation}
\noindent
\small {\bf Excercice 2.}
La formule de Stirling exprime,  pour $n$ grand, que
$$
n! \sim Cn^n\sqrt{n} \ \textrm{exp}(-n),
$$
o\`u $C=\sqrt{2\pi}$. Cette constante peut se calculer en utilisant la formule de Wallis,
que l'on trouve gr\^ace aux int\'egrales \'eponymes :
$$
\forall n\in N, I_n=\int_{0}^{\frac{\pi}{2}}(\sin x)^n dx.
$$
\noindent
\small {\bf Excercice 3.}
La fonction $\Gamma$: $R_{+}^{\star}\rightarrow R$, d\'efinie par
\begin{equation} \label{equation}
\Gamma(x)=\int_{0}^{+\infty}t^{x-1}e^{-t}dt
\end{equation}
la formule (\ref{equation}) est appel\'ee fonction Gamma (d'Euler), g\'en\'eralise la factorielle.\\
En effet, $\forall n \in N^{\star}, \Gamma(n+1)=n! $. On peut aussi montrer que
$$
\Gamma\left(\frac{1}{2}\right)=\sqrt{\pi},
$$
en se ramenant \`a l'in\'egrale de Gauss $I=\int_{0}^{+\infty}e^{-t^2}dt$
(par changement de variables), cette derni\`ere valant
$\frac{\sqrt{\pi}}{2}$ (par exemple en consid\'erant le carr\'e de $I$ et un passage en coordonn\'ees polaires).\\
\small {\bf Excercice 4.}
Ins\'erer le graphe de la fonction suivante dans un document Latex: \ \
$$
f(x)=x+e^x+\frac{10}{1+x^2}-5
$$
\noindent
\small {\bf Excercice 5.}
\`{A} savoir sur les m\'ethodes de quadratures ( voir le tableau \ref{tabm} )

\begin{table}
\begin{center}
\begin{tabular}{|l|r|}
\hline
Point milieu        &   1     \\
\hline
Trap\`ezes    &   1          \\
\hline
Simpson     &    3\\
\hline
\end{tabular}
\caption { m\'ethodes de quadratures \label{tabm}}
\end{center}
\end{table}

\end{document}
