\documentclass[a4paper,12pt]{article}
\usepackage[utf8]{inputenc}
\usepackage{fontenc}
\usepackage[french]{babel}
\usepackage{graphicx}
\usepackage{float}
\usepackage{amssymb}
\usepackage{amsmath}
\author{I.Djerrar, L.Alem, L. Chorfi}
\date{}
\title{TP4: Table des matières, figures, bibliographie.\\
Résolution d'un problème de la chaleur à symétrie radiale }
\bibliographystyle{}
\begin{document}
\maketitle % pour generer l'entent de l' article
\begin{abstract}
Dans ce travail on s'intéresse à un problème de la chaleur. On résout le problème direct qui servira à résoudre un problème inverse associé dans $\mathbb{R}.$
\end{abstract}
\tableofcontents
\newpage
\section{Position du problème}
On considère la problème de la théorie de la chaleur, on coordonnée radiale suivant
\begin{equation}
\begin{cases}
\frac{\partial u}{\partial t}=\frac{\partial ^2u}{\partial ^2 r}+\frac{1}{r}\frac{\partial u}{\partial r},\hspace{2cm} r\in\left( a,b\right), \quad t>0\\
u(a,t)=f(t) , \frac{\partial u}{\partial r }(b,t)=0 \hspace{2cm} t>0\\
u(r,0)=0, \hspace{3cm} r\in(a,b)
\end{cases}
\end{equation} 
Le but de ce travail est de résoudre ce problème par deux méthodes, analytique et approchée
\subsection{Construction de la solution à l'aide de la transformée de Laplace }
Soit $f$  une fonction tel que $|{f(t)}| \leq Ce^{\sigma T}, \sigma\geq 0,$\\
La transformée de la place $F(s)=L(f)$ est définie par:
$$F(s) = \int_{0}^{+\infty}f(t)e^{-st}dt,  Re(s)\geq \sigma $$
La transformée inverse est donnée par\cite{Ditkine}
$$f(t)= L^{-1}(F)(t)=\frac{1}{2\pi i}\int_{\sigma-i\infty}^{\sigma+i\infty}F(s)e^{st}ds$$
\subsection{Annexe}
Quelle que soit la valeur de $x$, la propriété suivante est toujours vérifie:
$$\sin^2x+\cos^2x=1$$
On peut s'en douter en observant le tracé de la fonction illustrée
\section{Un exemple avec référence bibliographique }
Dans ce travail on a utilise quelques références \footnote{Ces références sont dans la bibliothèque du département} et on a demendé à\footnote{}\cite{Ditkine} et \cite{Herbin}.
\section{Un exemple de tableau}
$$\begin{array}{|r|r|r|}
\hline n & x & y\\
\hline x_n & 5 & 8\\
\hline |e_n| & 0.05 &0.09\\
\hline x^2-1 & 1 & 8 \\
\hline
\end{array}$$ 
\section{Un exemple d'insertion d'un graphe d'une fonction}
Le graphe de la fonction $ g(x)=x^2+e^{x-1} $ sur $\left[ -1,1\right]$ est:
\begin{figure}[htbp]
\centering
\includegraphics[height=4cm]{graphe3.png}
\caption {\textit{graphe de la fonction $g(x)$.}}
\end{figure}
\begin{thebibliography}{}
\bibitem{Ditkine} Ditkine V. Proudnikov A. Transformation intégrales et calcul opérationnel. Traduit du russe edition MIR.Moscow, 1978.
\bibitem{Herbin} Herbin R., Analyse numérique des EDP. ENgineering school,Marseille 2011.
\end{thebibliography}{}
\end{document}
