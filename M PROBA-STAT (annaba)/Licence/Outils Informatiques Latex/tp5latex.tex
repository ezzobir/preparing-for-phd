\documentclass[11pt]{beamer}
\usepackage[utf8]{inputenc}
\usepackage[T1]{fontenc}
%\usepackage{lmodern}
\usepackage[french]{babel}
\usepackage{amsmath}
\usepackage{amsfonts}
\usepackage{amssymb}
\usepackage{graphicx}
\usetheme{warsaw}
	\title{Determining surface temperature for an axisymmetric inverse heat conduction problem}
	\author{\textsc{I.D}jerrar, \textsc{L.A}lem  and  \textsc{L.C}horfi}
		\institute{Lab. LMA, B.M University, Annaba 
		\\1ère édition du colloque National
		\\Mathématiques Appliquées et \'Equation Différentielles \\ "MAED'2018"" }
	\date{\textsc{N}ovember, 28, 2018,\textsc{E}l, \textsc{T}arf}
	\begin{document}
	\maketitle
	\begin{frame}
	\frametitle{The plan} \tableofcontents
	\end{frame}
	\begin{frame}
	\section{Introduction}
	\frametitle{Introduction}
	The inverse heat conduction problems \textcolor{magenta}{\textsc{(IHCB)}} arises from many physical and engineering problems. It is often impossible to directly measure the desired physical quantity. In this situation ,one way to measure this quantity is to solve a boundary value inverse heat conduction problem.\\
	 \qquad We willl consider an axisymmetric inverse problem for the heat aquation inside the cylinder   $ a \leq r \leq b$
	    
\end{frame}
\begin{frame}
\section{Direct problem}
		\frametitle{Direct problem}
We assume that $u(a,t)=f(t)$ is known and we consider the following direct problem:
\begin{equation}\label{eq1}
\left\lbrace 
\begin{array}{lrl}
\frac{\partial u}{\partial t}=\frac{\partial u^{2}}{\partial r^{2}}+\frac{1}{r}\frac{\partial u}{\partial r}, \ \ r\in (a,b),\ \ t\geq 0,\\
u(a,r)=f(t),\\
\frac{\partial u}{\partial r} (b,t)=0, \quad t\geq 0\\
u(r,0)=0 \quad r\in[a,b].
\end{array}
\right. 
\end{equation}
\end{frame}	
\begin{frame}
\subsection{Construction of the solution}
	\frametitle{construction of the solution}
	We use \textcolor{green}{the Laplace transform} for the representation of the solution:\\
	Let\\
\begin{equation}
		\textsc{F}=\mathcal{L}(\textsl{f})\Leftrightarrow \textsc{F}(s):=\int_{0}^{+\infty}e^{-st}f(t)dt
\end{equation}
The inverse \textcolor{violet}{Laplace transform} is given by the complex inversion formula\\
\begin{equation}
f(t)=\mathcal{L}^{-1}(\textsc{F})(t)=\frac{1}{2\pi i}\int_{\sigma-\infty}^{\sigma+\infty}e^{st}\textsc{F}(s)ds, \quad t>0 
\end{equation}
\end{frame}
\begin{frame}
\begin{theorem}
	\begin{itemize}
		\item Assume that $ f(t)\in C^{1}$,such that $f(0)=0$ for $t\geq$ T. Then the series $ (12)$ converge in $ L^{2}(]a,b)[)$ for all$ t\geq 0$ and defines one solution of the problem \eqref{eq1}  in $H.$
	\end{itemize}
\end{theorem}
\end{frame}
%\begin{frame}
%	\begin{figure}[h]
%		\centering
%		\includegraphics[height=4 cm]{untitled.png}
%		\caption{\textit{graphe une fonction.}}
%		\label{untitled}
%	\end{figure}
%\end{frame}
\begin{frame}
\section{Bibliograhy}
\frametitle{Bibliography}
\begin{thebibliography}{widestlabel}

	\bibitem{Kirsh}A. Kirsh. \textit{Introduction to mathematical theory of inverse problems}. Series AMS Vol. 120, Springer, 2011.
	\bibitem{Ditkine}A. Ditkine, A. Proudnikov.\textit{Transformation intégrales et calcul opérationnel} Traduit du russe, edition MIR,Moscou,1978.
	\bibitem{Cheng} W.Cheng. Regularization and stability estimates for an inverse source problem of the radially symmetric parabolic equation. \textit{Journal of Inequalities and Applications}, 2015:136, 2015.
	\bibitem{Cheng}W. Cheng, C-L. Fu. Tow regularization methods for an axisymmetric inverse heat conduction problem. \textit{J. Inv. Ill-Posed Problems}, 17, 159-172, 2009.
\end{thebibliography}
\end{frame}
\end{document}
