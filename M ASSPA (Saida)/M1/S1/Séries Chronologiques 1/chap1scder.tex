\documentclass[a4paper,12pt,freqn]{report}
%\usepackage[latin1]{inputenc}
\usepackage{mathenv}
\usepackage[english]{babel}
\usepackage[T1]{fontenc}
\usepackage{amssymb}
\usepackage{systeme}
\usepackage{amsthm}
\usepackage{amsmath}
\usepackage{bbding}
\usepackage{array}
\usepackage{dsfont}
\usepackage{frcursive}
\usepackage{niceframe}
\usepackage{calligra}
%\usepackage{chancery}
\usepackage{niceframe}
\usepackage{lettrine}
\usepackage{graphicx}
\usepackage{newlfont}
\usepackage{amsfonts}
\usepackage{stmaryrd}
\pagestyle{empty}
\usepackage{color}
\linespread{1.56}
\usepackage{mathtools}
\usepackage{hyperref}
%\usepackage{top=2cm,bottom=2cm,right=2cm,left=2cm}
\usepackage[nottoc,notbib]{tocbibind}
%\usepackage[ps2pdf, colorlinks=true, linkcolor=blue, urlcolor=green, pdfstartview=FitH, pdfhighlight =/O]{hyperref}
\newlength{\plarg}
\setlength{\plarg}{17cm}
\def\zcdefault{zc}
\DeclareRobustCommand\zcfamily{%
 \fontfamily\zcdefault\selectfont
 }
 \DeclareTextFontCommand{\textzc}{\zcseries}
%\pagestyle{headings}
\usepackage{fancyhdr}
\pagestyle{fancy}
\renewcommand{\chaptermark}[1]{\markright{#1}{}}
\renewcommand{\sectionmark}[1]{\markright{\thesection\ #1}}
\renewcommand{\subsectionmark}[1]{\markright{\thesubsection\ #1}}
%\fancyhf{} \fancyhead[LE,RO]{\bfseries\thepage}
%\fancyhead[LO]{\bfseries\rightmark}
%\fancyhead[RE]{\bfseries\rightmark}
\makeatletter
\newcommand\tcaption{\def\@captype{table}\caption}
\newcommand\fcaption{\def\@captype{figure}\caption}
\makeatother

\renewcommand{\headrulewidth}{0.5pt}
\renewcommand{\footrulewidth}{0pt}
%\addtolength{\headheight}{0.5pt} \fancypagestyle{headings}{
%\fancyhead{}
%\renewcommand{\headrulewidth}{0pt}
%} 
\theoremstyle{plain}
\def\og{\leavevmode\raise.3ex\hbox{$\scriptscriptstyle\langle\!\langle$~}}
\def\fg{\leavevmode\raise.3ex\hbox{~$\!\scriptscriptstyle\,\rangle\!\rangle$}}
%\renewcommand{\baselinestretch}{1.5}
\renewcommand{\baselinestretch}{1.2}
\newcommand{\cqfd}{\hfill \rule{2mm}{2mm}}
\def\as={\,\stackrel{a.s.}{=}\,} %%%%%%%%%%%%%%%%%%%%%%%%%%%%%
%notations symetriques
\newcommand{\R}{\mathbb R} %ensemble des reels
\newcommand{\Q}{\mathbb Q} %ensemble des rationnels
\newcommand{\C}{\mathbb C} %ensemble des complexes
\newcommand{\Z}{\mathbb Z} %ensemble des relatifs
\newcommand{\N}{\mathbb N} %ensemble des entiers
\newcommand{\p}{\mathbb P} %probabilite
\newcommand{\E}{\mathbb E} %

\def\1{{1\!\!1}}
\def\P{{\mathrm {I\!P}}}
\def\E{{\mathrm {I\!E}}}
\newdimen\AAdi%
\newbox\AAbo%
\font\AAFf=cmex10%   %ou cmex10
\def\AArm{\fam0 \rm}%
\def\AAk#1#2{\setbox\AAbo=\hbox{#2}\AAdi=\wd\AAbo\kern#1\AAdi{}}%
\def\AAr#1#2#3{\setbox\AAbo=\hbox{#2}\AAdi=\ht\AAbo\raise#1\AAdi\hbox{#3}}%
%% alphabet blackboard
%% 1 blackboard
\def\1{{\mathrm {1\AAk{-.8}{I}I}}}%
\def\cov{\hbox{\it Cov}}
\def\loinorm{\hbox{\cal N}}
\def\card{\hbox{\it card}}
\def\var{\hbox{\it Var}}
\def\P{{ \hbox{\sc I\hskip -2pt P}}} %r‚els
\def\R{{ \hbox{\sc I\hskip -2pt R}}} %r‚els
\def\N{{ \hbox{\sc I\hskip -2pt N}}} %naturels
\def\Z{{\bf Z}} %entiers relatifs
\def\Q{{\bf \hbox{\sc I\hskip -7pt Q}}} %rationnels
\def\C{{\bf \hbox{\sc I\hskip -7pt C}}} %complexe
\def\I{{\bf \hbox{\sc I\hskip -2pt I}}} %complexe
\def\F{\widehat{F }}
\def\f{\widehat{f }}
\def\h{\widehat{h }}
\providecommand{\abs}[1]{\lvert#1\rvert}
\def\Th{th\'eor\`eme}
\newtheorem{theor}{Theorem}[section]
\newtheorem{Theor}{Theorem}[subsection]
\newtheorem{theorem}{Theorem}[chapter]
\newtheorem{theoreme}{Theorem}[section]
\newtheorem{theo}{Theorem}[subsection]
\newtheorem{theore}{Theorem}[subsubsection]
\newtheorem{corollaire}{Corollary}[chapter]
\newtheorem{coro}{Corollary}[section]
\newtheorem{corolla}{Corollary}[section]
\newtheorem{cor}{Corollary}[subsection]
\newtheorem{coroll}{Corollary}[subsubsection]
\newtheorem{llem}{Lemma}[chapter]
\newtheorem{lem}{Lemma}[section]
\newtheorem{lemme}{Lemma}[subsection]
\newtheorem{lemm}{Lemma}[subsubsection]
\newtheorem{proposition}{Proposition}[section]
\newtheorem{prop}{Proposition}[subsection]
\newtheorem{pr}{Proposition}[chapter]
\newtheorem{propos}{Proposition}[subsubsection]
\newtheorem{definition}{Definition}[section]
\newtheorem{defn}{Definition}[subsection]
\newtheorem{defne}{Definition}[subsection]
\newtheorem{defntn}{Definition}[subsubsection]
\newtheorem{rem}{Remark}[section]
\newtheorem{re}{Remark}[subsection]
\newtheorem{rema}{Remark}[subsection]
\newtheorem{remar}{Remark}[subsubsection]
\newtheorem{exe}{Example}
\newtheorem{propr}{Propriété}
\newtheorem{dem}{Démonstration}
\def\bt{\begin{theorem}}
\def\et{\end{theorem}}
\makeatletter
\def\clap#1{\hbox to 0pt{\hss #1\hss}}%
\def\ligne#1{%
\hbox to \hsize{%
\vbox{\centering #1}}}%
\def\haut#1#2#3{%
\hbox to \hsize{%
\rlap{\vtop{\raggedright #1}}%
\hss
\clap{\vtop{\centering #2}}%
\hss
\llap{\vtop{\raggedleft #3}}}}%
\def\bas#1#2#3{%
\hbox to \hsize{%
\rlap{\vbox{\raggedright #1}}%
\hss
\clap{\vbox{\centering #2}}%
\hss
\llap{\vbox{\raggedleft #3}}}}%
\def\maketitle{%
\thispagestyle{empty}\vbox to \vsize{%
\haut{}{\@blurb}{} \vfill \vspace{1cm}
\begin{flushleft}
\usefont{OT1}{ptm}{m}{n}
\huge \@title
\end{flushleft}
\par
\hrule height 4pt
\par
\begin{flushright}
\usefont{OT1}{phv}{m}{n}
\Large \@author
\par
\end{flushright}
\vspace{1cm} \vfill \vfill \bas{}{\@location, }{}
}%
\clearpage }
%\def\date#1{\def\@date{#1}}
%\def\author#1{\def\@author{#1}}
%\def\title#1{\def\@title{#1}}
%\def\location#1{\def\@location{#1}}
%\def\blurb#1{\def\@blurb{#1}}
\def\ds{\displaystyle}

%\clearpage\hypersetup{
   % colorlinks=true, %colorise les liens
    %breaklinks=true, %permet le retour à la ligne dans les liens trop longs
    %urlcolor= blue,  %couleur des hyperliens
    %linkcolor= blue, %couleur des liens internes
    %plainpages=false  %pour palier à "Bookmark problems can occur when you have duplicate page numbers, for example, if you have a page i and a page 1."
%}
\begin{document}



\chapter{Concepts des s\'{e}ries tomporelles}
\section{Introduction}
\subsection{ Vocabulaire et exemples}
\begin{defi}
Une suite d'observations d'une famille de variables al\'{e}atoires r\'{e}elles $(X_{t})_{t\in\Theta}$ index\'{e}e par le temps est appel\'{e}e s\'{e}rie
chronologique (ou temporelle).
$o\grave{u}$ l'ensemble $\Theta$ est appel\'{e} espace de temps qui peut \^{e}tre :\\
\begin{itemize}
  \item \textbf{discret} (nombre d'\'{e}tudiants inscrits \`{a} l'universit\'{e}, temp\'{e}rature maximale, nombre mensuel de vente de voitures en Alg\`{e}rie, nombre annuel de naissance en Alg\`{e}rie, ...). Dans ce cas, $\Theta\subset\mathbb{Z}$.
Les dates d'observations sont le plus souvent \'{e}quidistantes : par exemple relev\'{e}s mensuels, trimestriels...Ces dates \'{e}quidistantes sont alors index\'{e}es par des entiers : t = 1, 2, . . . ,T et T est le nombre
d'observations. On dispose donc des observations des variables $X_{1},X_{2},...,X_{T}$. Ainsi si h est l'intervalle de temps s\'{e}parant deux
observations et $t_{1}$ l'instant de la premi\`{e}re observation, on a le sch\'{e}ma suivant:
\begin{center}
    $t_{1}$ \ $t_{1}+h$  $\ldots$  $t_{1}+(T-1)h$ \\
     $X_{t_{1}}$  \  $X_{t_{1
     +h}}$ $\ldots$   $X_{t_{1}+(T-1)h}$ \\
     $X_{1}$ \  $X_{2}$  $\ldots$  $X_{T}$
\end{center}

  \item \textbf{continu} (signal radio, r\'{e}sultat d'un \'{e}lectrochardiogramme...). L'indice de temps est \`{a} valeurs dans
un intervalle de $\mathbb{R}$ et on dispose (au moins potentiellement) d'une infinit\'{e} d'observations issues
d'un processus $(X_{t})_{t\in\Theta}$ $o\grave{u}$ $\Theta$ est un intervalle de  $\mathbb{R}$. Un tel processus est dit \`{a} temps continu.
\end{itemize}
\end{defi}
\begin{Remark} 
 Les dates d\`{}observations sont g\`{e}n\`{e}ralement ordonn\`{e}es de manière r\`{e}gulière dans le temps : on manipule des s\`{e}ries
 \begin{itemize}
   \item  journali\`{e}res (cours d\`{}une action en bourse).
   \item  mensuelles (consommation mensuelle d\`{}\`{e}lectricit\`{e}).
   \item  trimestrielles (nombre trimestriel de ch\^{o}meurs).
   \item  annuelles (chiffre annuel des b\`{e}n\`{e}fices des exportations).
 \end{itemize}
\end{Remark}


annuelles (chiffre annuel des b\`{e}n\`{e}fices des exportations).

\subsection{Exemples}
On peut songer par exemple \`{a} : \\
1- l'\'{e}volution du nombre mensuel de voyageurs utilisant le  transport a\'{e}rien, (s\`{e}rie AirPassengers du langage R)\\

\begin{tabular}{|c|c|c|c|c|c|c|c|c|c|c|c|c|}
  \hline
  % after \\: \hline or \cline{col1-col2} \cline{col3-col4} ...

     & Jan& Feb &Mar &Apr &May &Jun &Jul &Aug &Sep &Oct &Nov &Dec \\

1949 &112& 118& 132& 129& 121 &135 &148 &148 &136 &119 &104 &118 \\

1950 &115 &126 &141 &135 &125 &149 &170 &170 &158 &133 &114 &140 \\

1951 &145 &150 &178 &163 &172 &178 &199 &199 &184 &162 &146 &166 \\

1952 &171 &180 &193 &181 &183 &218 &230 &242 &209 &191 &172 &194 \\

1953 &196 &196 &236 &235 &229 &243 &264 &272 &237 &211 &180 &201 \\

1954 &204 &188 &235 &227 &234 &264 &302 &293 &259 &229 &203 &229 \\

1955 &242 &233 &267 &269 &270 &315 &364 &347 &312 &274 &237 &278 \\

1956 &284 &277 &317 &313 &318 &374 &413 &405 &355 &306 &271 &306 \\

1957 &315 &301 &356 &348 &355 &422 &465 &467 &404 &347 &305 &336 \\

1958 &340 &318 &362 &348 &363 &435 &491 &505 &404 &359 &310 &337 \\

1959 &360 &342 &406 &396 &420 &472 &548 &559 &463 &407 &362 &405 \\

1960 &417 &391 &419 &461 &472 &535 &622 &606 &508 &461 &390 &432 \\
\hline
\end{tabular}
\bigskip

% Requires \usepackage{graphicx}
  %\includegraphics[width]{air.pdf}\\
  %\caption{Figure 1}\label{}
  \begin{center}
  \includegraphics[width=10cm]{air.pdf}\\


  \textbf{Figure 1.1}
\end{center}


2- la temp\'{e}ratures mensuelles moyennes \`{a} Nottingham, 1920-1939 :


\begin{center}
  % Requires \usepackage{graphicx}
  \includegraphics[width=10cm]{nottem.pdf}\\
  \textbf{Figure 1.2}
\end{center}

3-la concentration atmosph\'{e}rique de CO2 \`{a} Maunaloa :


\begin{center}
  % Requires \usepackage{graphicx}
  \includegraphics[width=10cm]{Co2.pdf}\\
\textbf{Figure 1.3}
\end{center}
4- \ l'accroissement relatif mensuel de l'indice des prix.

\subsection{Domaines d\`{}application}
On trouve des exemples de s\'{e}ries chronologiques univari\'{e}es dans de tr\'{e}s nombreux domaines.
La liste suivante n\'{}est qu\'{}un \'{e}chantillon:
\begin{itemize}

  \item finance et \'{e}conom\'{e}trie : \'{e}volution des indices boursiers, des prix, des donn\'{e}es \'{e}conomiques
des entreprises, des ventes et achats de biens, des productions agricoles ou industrielles,
  \item assurance : analyse des sinistres,

  \item  m\'{e}decine / biologie : suivi des \'{e}volutions des pathologies, analyse d\'{}\'{e}lectro-enc\'{e}phalogrammes
et d\'{}\'{e}lectrocardiogrammes,

  \item sciences de la terre et de l\'{}espace : indices de mar\'{e}es, variations des ph\'{e}nom\'{e}nes physiques
(m\'{e}t\'{e}orologie), \'{e}volution des taches solaires, ph\'{e}nom\'{e}nes d\'{}avalanches,
  \item traitement du signal : signaux de communications, de radars, de sonars, analyse de la parole,

  \item traitement des donn\'{e}es : mesures successives de position ou de direction d\'{}un objet mobile (trajectographie).
\end{itemize}


\subsection{Objectifs principaux}
L'\'{e}tude d'une s\'{e}rie chronologique permet \textbf{d'analyser, de d\'{e}crire et d'expliquer} un ph\'{e}nom\`{e}ne au cours
du temps et d'en tirer des cons\'{e}quences pour des prises de d\'{e}cision (marketing...). \\
\\ Cette \'{e}tude permet aussi de faire un contr$\hat{o}$le, par exemple pour le gestion des stocks, le contr$\hat{o}$le d'un
processus chimique.\\
Mais l'un des objectifs principaux de l'\'{e}tude d'une s\'{e}rie chronologique est la pr\'{e}vision qui consiste \`{a}
pr\'{e}voir les valeurs futures $X_{T+h}$ (h = 1, 2, 3, . . .) de la s\'{e}rie \`{a} partir de ses valeurs observ\'{e}es
jusqu'au temps T: $X_{1}$,\ $X_{2}$,...,\ $X_{T}$ .\\
Il existe encore bien d'autres objectifs imm\'{e}diats relatifs \`{a} l'\'{e}tude des s\'{e}ries chronologiques. Par exemple : \\

 si deux s\'{e}ries sont observ%
 \'{e}es, on peut se demander quelle \textbf{influence elles exercent l'une
sur l'autre}. En notant X$_{t}$ et Y$_{t}$ les deux s\'{e}ries en question,
on examine s'il existe par exemple des relations du type

Y$_{t}$ = a$_{1}$X$_{t-1}$ + a$_{3}$X$_{t-3}$.

Ici, deux questions se posent : tout d'abord, la question de \textbf{la
causalit\'{e}} i.e. quelle variable (ici (X$_{t}$)) va expliquer l'autre
(ici (Y$_{t}$)), ce qui am\`{e}ne \`{a} la deuxi\`{e}me question, celle \textbf{du
d\'{e}calage temporel} : si une influence de (X$_{t}$) sur (Y$_{t}$) existe,
avec quel d\'{e}lai et pendant combien de temps la variable explicative (X$%
_{t}$) influence-t-elle la variable expliqu\'{e}e (Y$_{t}$) ?

\mathbf{\textbf{Application}}
la s\'{e}rie suivante repr\'{e}sente le nombre de crimes par trimestre d'une certaine ville  $17-20-30-15-19-24-36-20-22-27-39-23-25-30-42-26$
\begin{itemize}
  \item Tracer le graphe de cette s\'{e}rie.
\end{itemize}

\bigskip

\bigskip
\newpage
\section{\textbf{ Description d'une s\'{e}rie chronologique}}

On consid\`{e}re qu'une s\'{e}rie chronologique (X$_{t}$) est la r\'{e}%
sultante de diff\'{e}rentes composantes fondamentales :

\bigskip

\subsection{  la tendance ou trend}
 (Z$_{t}$) repr\'{e}sente l'%
\'{e}volution \`{a} long terme de la s\'{e}rie \'{e}tudi\'{e}e. Elle traduit le
comportement \textquotedblright moyen\textquotedblright\ de la s\'{e}rie. Elle est le plus souvent mod\'{e}lis\'{e}e par une fonction lin\'{e}aire ou
polynomiale du temps.
Par exemple, la s\'{e}rie de la Figure 1.3 a tendance \`{a} augmenter de fa\c{c}on
lin\'{e}aire et indique que le nombre de voyageurs a augment\'{e} de mani\`{e}re r\'{e}%
guli\`{e}re au cours du temps.

\bigskip

\subsection{la composante saisonni\`{e}re (ou saisonnalit\'{e})}


$S_{t}$ correspond \`{a} un ph\'{e}nom\`{e}ne qui se r\'{e}p\`{e}te \`{a}
intervalles de temps r\'{e}guliers (p\'{e}riodiques). En g\'{e}n\'{e}ral,
c'est un ph\'{e}nom\`{e}ne saisonnier d'\`{o}u le terme de variations saisonni%
\`{e}res.
Les variations saisonni\`{e}res sont li\'{e}es au rythme impos\'{e} par les
saisons m\'{e}t\'{e}orologiques (production agricole, consommation de gaz,
vente des cr\`{e}mes solaires. . . ) ou encore par des activit\'{e}s \'{e}%
conomiques et sociales ( f\^{e}tes, vacances, soldes,. . . ).
 C'est un ph\'{e}nom\`{e}ne qui
se reproduit de mani\`{e}re analogue sur chaque intervalle de temps
successif. Lorsqu'on veut mettre en \'{e}vidence ce ph\'{e}nom\`{e}ne \`{a}
l'aide d'un graphique, on peut d\'{e}couper la s\'{e}rie en sous-s\'{e}ries
de longueur de p\'{e}riode P du saisonnier et repr\'{e}senter ces sous-s\'{e}%
ries sur un m\`{e}me graphique. Dans la suite. La plupart du temps, on suppose que la composante saisonni%
\`{e}re est constante sur chaque p\'{e}riode P, c'est-`a-dire S$_{t+P}$ = S$%
_{t}$.
0
Lorsque P = 4, la s\'{e}rie est trimestrielle ; lorsque P = 12, la s\'{e}rie
est mensuelle. (cf. Figure1.4). Sur ce
graphique, on voit bien une similarit\'{e} des diff\'{e}rentes courbes
annuelles li\'{e}e aux saisons \ m\'{e}t\'{e}orologiques.\\

\subsection{la composante r\'{e}siduelle. (ou bruit ou r\'{e}sidu)}



$Y_{t}$ correspond \`{a} des fluctuations irr\'{e}guli\`{e}res, en g\'{e}n\'{e}%
ral de faible intensit\'{e} mais de nature al\'{e}atoire. On parle aussi d'al%
\'{e}as. sont des variables al\'{e}atoires centr\'{e}es. On
consid\`{e}re le plus souvent un bruit blanc, c'est-\`{a}-dire une suite de
v.a.r. telles que
\begin{array}{ll}
     E($\varepsilon _{t}$) = 0 et E($\varepsilon _{t1}\varepsilon _{t2}$) = & \left\{
                                                  \begin{array}{l1}
                                                      $\sigma^{2}$ &  si\ $t1=t2$ \\
                                                       0   & sinon
                                                  \end{array}
                                                \right.
       \\
 \end{array}$

Les v.a.r. sont alors non corr\'{e}l\'{e}es et lorsque le bruit blanc est
gaussien c'est-`a-dire que $\varepsilon _{t}$ \sim N(0, \sigma$^{2}$),
on a de plus l'ind\'{e}pendance des $\varepsilon _{t}$.


 Par exemple, la s\'{e}rie  de la Figure 1.4 a un comportement
assez irr\'{e}gulier : il y a comme une sorte de bruit de faible amplitude
qui perturbe les donn\'{e}es.

Les mod\`{e}les pr\'{e}sent\'{e}s dans \`{a} la suite tiennent compte de ces trois
composantes (tendance, saisonnalit\'{e} et fluctuations irr\'{e}guli\`{e}%
res). Il faut cependant remarquer que l'on pourrait envisager d'autres
composantes.

\subsection{ Des ph\'{e}nom\`{e}nes accidentels}
 gr\`{e}ves,
conditions m\'{e}t\'{e}orologiques exceptionnelles, crash financier) peuvent
notamment intervenir. Par exemple, la s\'{e}rie d) de la Figure 1. pr\'{e}%
sente deux cassures.


\section{ Mod\'{e}lisation}

Un mod\`{e}le est une image simplifi\'{e}e de la r\'{e}alit\'{e} qui vise \`{a} traduire les m\'{e}canismes de fonctionnement du ph\'{e}nom\`{e}ne
\'{e}tudi\'{e}. On distingue principalement deux types de mod\`{e}les :

\subsection{les mod\`{e}les d\'{e}terministes.}

\textbf{\ }X$_{t}$ = f(t,$\varepsilon _{t}$).

Les deux mod\`{e}les de ce type les plus usit\'{e}s sont les suivants

\subsubsection{le mod\`{e}le additif}.


X$_{t}$ s'%}
%TCIMACRO{\U{b4}}%
%BeginExpansion
\'{}%
%EndExpansion
ecrit comme le somme de trois termes : \\
 $X_{t} = Z_{t} + S_{t} + \varepsilon _{t}$,

\subsubsection{le mod\`{e}le multiplicatif}.
 La variable X$_{t}$ s'\'{}ecrit \\
$X_{t}= Z_{t} \times S_{t} \times\varepsilon _{t}$ .

\subsection{ les mod\`{e}les stochastiques}.

Ils sont du m\`{e}me type que}
les mod\`{e}les d\'{e}terministes \`{a} ceci pr\'{e}s que les variables de
bruit $\varepsilon _{t}$ ne sont pas iid mais poss\`{e}dent une structure de
corr\'{e}lation non nulle : $\varepsilon _{t}$ est une fonction des valeurs
pass\'{e}es ( lointaines suivant le mod\`{e}le) et d'un terme d'erreur
\eta_{t}

$\varepsilon _{t}$ = g($\varepsilon _{t-1}$, $\varepsilon _{t-2}$, . . . ,
\eta_{t}).

La classe des mod\`{e}les de ce type la plus fr\'{e}quemment utilis\'{e}e est
la classes des mod\`{e}les SARIMA (et de ses sous-mod\`{e}les ARIMA, ARMA,...).

Les deux types de mod\`{e}les ci-dessus induisent des techniques de pr\'{e}%
vision bien particuli\`{e}res.
\section{Mod\`{e}lisation d\`{e}terministe}
\subsection{ le mod\`{e}le additif}
X$_{t}$ = Z$_{t}$ + S$_{t}$ + $\varepsilon _{t}$, t\in Z.

-- Dans ce mod\`{e}le, l'amplitude de la s\'{e}rie reste constante au cours
du temps. Cela se traduit graphiquement par des fluctuations autour de la
tendance Z$_{t}$ constantes au bruit pr\`{e}s.\\
\textbf{Remarque 2.1 }
-- A premi\`{e}re vue, la notion de composante p\'{e}riodique pourrait
\^{e}tre suffisante. Cependant, ce n'est pas le cas pour la raison d\'{e}%
crite ci-apr\`{e}s. Consid\'{e}rons le mod\`{e}le additif

X$_{t}$ = Z$_{t}$ + S$_{t}$ + $\varepsilon _{t}$, t\in Z.

Cette d\'{e}composition n'est pas unique en l'absence d'hypoth\`{e}ses suppl%
\'{e}mentaires. En effet, si S$_{t}$ est une composante p\'{e}riodique de p%
\'{e}riode p, alors il existe une constante c telle que

S$_{t+1}$ + . . . + S$_{t+p}$ = c, t\in Z.

Dans ce cas, on a aussi la d\'{e}composition suivante

$X_{t} =  \grave{Z}_{t} + \grave{S}_{t}+ \varepsilon _{t},$

\grave{ou}  pour tout t $\in Z, \grave{Z}_{t} = Z_{t}+ c/p$ et $\grave{S}_{t} = S_{t}-
c/p $. On a donc trouv\'{e} une autre d\'{e}composition du signal X$_{t}$. On
remarquera que $\grave{S}_{t}$ est une composante de somme nulle sur la p\'{e}riode
p. C'est pourquoi on impose \`{a} toute composante saisonni\`{e}re d\`{}etre p\'{e}riodique et de somme nulle sur une p\'{e}riode$(\frac{1}{p}\sum^{p}_{i=1}S_{i}=0)$.\\
  % Requires \usepackage{graphicx}
 \begin{center}
  % Requires \usepackage{graphicx}
  \includegraphics[width=10cm]{dddd.pdf}\\
\textbf{Figure 1.4}
\end{center}



\subsection{le multiplicatif}

Nous consid\'{e}rons dans cette section une s\'{e}rie $X = (X_{t})_{t}$
admettant une d\'{e}composition muliplicative

X$_{t}$ = Z$_{t}$\times $S_{t}$\times$\varepsilon _{t}$, t = 1 . . . T,

Dans ce cas l'amplitude de la s\'{e}rie n'est
plus constante au cours du temps : elle varie au cours du temps
proportionnellement \`{a} la tendance Z$_{t}$ au bruit pr\`{e}s. Dans ce mod%
\`{e}le, on suppose que $ \frac{1}{p}\sum^{p}_{i=1}S_{i}=1$.

\bigskip
\begin{Remark}\begin{itemize}
                \item Dans le cas d\`une s\`{e}rie $(X_{t})$ \grave{à} valeurs positives, ce 2e mod\`{e}le multiplicatif se ram\'{e}ne à un mod\'{e}le additif en consid\`{e}rant la s\`{e}rie $(ln(X_{t})$ :$ ln(X_{t}) = ln(C_{t}) + ln(S_{t}) + ln(ε_{t})$. La seule diff\`{e}rence entre les 2 mod\'{e}les multiplicatifs est dans lestimation des $ε_{t}$ qui n\`{}a pas une grande importance.
                \item  Le mod\`{e}le multiplicatif est g\'{e}n\'{e}ralement
utilis\'{e} pour des donn\'{e}es de type \'{e}conomique.
              \end{itemize}

\end{Remark}

  % Requires \usepackage{graphicx}
  \includegraphics[width=10cm]{multi.pdf}\\
  \caption{}\label{}



\subsection{Les modeles mixtes}}

Il s'agit  de mod\`{e}les ou addition et multiplication sont utilis\'{e}%
es. On peut supposer par exemple que la composante saisonni\`{e}re agit de
facon multiplicative alors que les fluctuations irr\'{e}gulieres sont
additives :

X$_{t}$ = Z$_{t}$ S$_{t}$ + $\varepsilon _{t}$, t = 1 . . . T.
\end{document}
